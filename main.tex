\def\year{2018}\relax
%File: formatting-instruction.tex
\documentclass[letterpaper]{article} %DO NOT CHANGE THIS
\usepackage{aaai18}  %Required
\usepackage{times}  %Required
\usepackage{helvet}  %Required
\usepackage{courier}  %Required
\usepackage{url}  %Required
\usepackage{graphicx}  %Required
%Non aaai-required
\usepackage{booktabs}
\usepackage{cleveref}
\usepackage{xcolor} 
\usepackage{eso-pic}
\usepackage{xspace}
\usepackage{float}
\usepackage{tcolorbox}
\usepackage{epsfig}
\usepackage{amsmath}
\usepackage{amssymb}
\usepackage{color}
\usepackage{tabu}
\usepackage{multirow}
\usepackage{pifont}% http://ctan.org/pkg/pifont
\usepackage{tikz}
\usepackage{pgfplots}
\usepackage{tikz}
\frenchspacing  %Required
\setlength{\pdfpagewidth}{8.5in}  %Required
\setlength{\pdfpageheight}{11in}  %Required
%PDF Info Is Required:
  \pdfinfo{
/Title (Speech-Based Visual Question Answering)
/Author (Anonymous Authors)}
\setcounter{secnumdepth}{0}
\usepgfplotslibrary{external}
\usetikzlibrary{positioning}
\graphicspath{{./images/}}
\pgfplotsset{compat=1.13}

\newcommand*{\ityping}{\includegraphics[scale=0.02]{7.png}}
\newcommand*{\ispeaking}{\includegraphics[scale=0.03]{6.png}}
\newcommand{\YES}{\ding{51}}
\newcommand{\NO}{\ding{55}}
\newcommand{\todo}[1]{\textcolor{blue}{#1}}
\newcommand{\vect}[1]{\mathbf{#1}}
\DeclareMathOperator*{\Unif}{U}
\newcommand{\tz}[1]{\textcolor{purple}{\textit{#1}}}
\begin{document}
\title{Speech-Based Visual Question Answering}
%%% Commented out for Arxiv submission
%\titlenote{Produces the permission block, and
%  copyright information}
%\subtitlenote{The full version of the author's guide is available as
%  \texttt{acmart.pdf} document}

\author{Ted Zhang\\
KU Leuven\\
tedz.cs@gmail.com}

% \author{Dengxin Dai}
% \affiliation{%
%   \institution{ETH Zurich}
% }
% \email{dai@vision.ee.ethz.ch}

% \author{Tinne Tuytelaars}
% \affiliation{%
%   \institution{KU Leuven}
% }
% \email{tinne.tuytelaars@esat.kuleuven.be }

% \author{Marie-Francine Moens}
% \affiliation{%
%   \institution{KU Leuven}
% }
% \email{sien.moens@cs.kuleuven.be}

% \author{Luc Van Gool}
% \affiliation{%
%   \institution{ETH Zurich}
% }
% \email{vangool@vision.ee.ethz.ch}
\maketitle
\begin{abstract}
This paper introduces speech-based visual question answering (VQA), the task of generating an answer given an image and a spoken question. Our work is the first study of speech-based VQA with the intention of providing insights for applications such as speech-based virtual assistants. Two methods are studied: an end to end, deep neural network that directly uses audio waveforms as input versus a pipelined approach that performs ASR (Automatic Speech Recognition) on the question, followed by text-based visual question answering. Our main findings are 1) speech-based VQA achieves slightly worse results than the extensively-studied VQA with noise-free text and 2) the end-to-end model is competitive even though it has a simple architecture. Furthermore, we investigate the robustness of both methods by injecting various levels of noise into the spoken question and find both methods to be tolerate noise at similar levels.
\tz{DO NOT USE LONGCITE EVERYWHERE}
\end{abstract}


\section{Introduction}
The recent years have witnessed great advances in computer vision, natural language processing, and speech recognition thanks to the advances in deep learning \cite{lecun2015deep} and abundance of data \cite{imagenet:2015}. This is evidenced not only by the surge of academic papers, but also by the world-wide industry interests. The convincing successes in these individual fields naturally raise the potentials of further integration towards solutions to more general AI problems. Much work has been done to integrate vision and language, resulting in a wide collection of successful applications such as image/video captioning \cite{show:tell:caption}, movie-to-book alignment \cite{align:bookmovie}, and visual question answering (VQA) \cite{VQA}. However, the importance of integrating vision and speech has remained relatively unexplored.

Pertaining to practical applications, voice-user interface (VUI) has become more commonplace, and people are increasingly taking advantage of its characteristics; it is natural, hands-free, eyes-free, far more mobile and faster than typing on certain devices \cite{speech:faster}. As many of our daily tasks are relevant to visual scenes, there is a strong need to have a VUI to talk to pictures or videos directly, be it for communication, cooperation, or guidance. Speech-based VQA can be used to assist blind people in performing ordinary tasks, and to dictate robotics in real visual scenes in a hand-free manner such as clinical robotic surgery.


%Diagram
\begin{figure}[t]
\centering
\begin{tikzpicture}[
bluesquare/.style={rectangle, draw=blue!70, fill=blue!4, very thick, minimum size=5mm},
greensquare/.style={rectangle, draw=green!70, fill=green!4, very thick, minimum size=5mm},
magsquare/.style={rectangle, draw=magenta!70, fill=magenta!4, very thick, minimum size=5mm},
empty/.style={rectangle, very thick, minimum size=7mm},
]

%Nodes TextMod
\node[magsquare, outer sep=5pt]      (model2)           {TextMod};
\node[empty] (answer2) [right=of model2] {pizza};
\node[inner sep=5pt] (pizza2)  [above=of model2] {\includegraphics[width=.1\textwidth]{pizzas.jpg}};
\node[greensquare, outer sep=5pt, label=below:what food is this?]    (asr)  [left=1cm of model2] {ASR}; 
\node[inner sep=5pt] (wav2)  [above=of asr] {\includegraphics[width=.1\textwidth]{wavs.png}}; 

%Nodes SpeechMod
\node[inner sep=5pt] (pizza1)  [below=of model2] {\includegraphics[width=.1\textwidth]{pizzas.jpg}};
\node[bluesquare, outer sep=5pt]   [below=of pizza1]   (model1)                      {SpeechMod};
\node[empty] (answer1) [right=of model1] {pizza};
\node[inner sep=5pt] (wav1)  [left=of model1] {\includegraphics[width=.1\textwidth]{wavs.png}};


% %Lines
\draw[->] (wav1.east) -- (model1.west);
\draw[->] (pizza1.south) -- (model1.north);
\draw[->] (model1.east) -- (answer1.west);

\draw[->] (wav2.south) -- (asr.north);
\draw[->] (asr.east) -- (model2.west);
\draw[->] (pizza2.south) -- (model2.north);
\draw[->] (model2.east) -- (answer2.west);

\end{tikzpicture}
\caption{An example of speech-based visual question answering and the two method in this study. A spoken question \emph{what food is this?} is asked about the picture, and the system is expected to generate the answer \emph{pizza}. TextMod uses a pipelined approach with ASR to retrieve the text first, while SpeechMod utilizes audio inputs directly.}
\label{fig:models}
\end{figure}

This work investigates the potential of integrating vision and speech in the context of VQA. A spoken version of the \textit{VQA1.0} dataset is generated to study two different methods of speech-based question answering. One method is an end-to-end approach based on a deep neural network architecture, and the other uses an ASR to first transcribe the text from the spoken question, as shown in \Cref{fig:models}. The former approach is particularly useful for languages that are not serviced by popular ASR systems, i.e. minor languages that have scarce text-speech aligned training data.

The \textbf{main contributions} of this paper are three-fold: \textbf{1)} We introduce an end-to-end model that directly produces answers from auditory input, without transformations into intermediate pre-learned representations, and compare this with a pipelined approach that first converts speech to text. \textbf{2)} We inspect the performance impact of having different levels of background noise mixed with the original utterances. \textbf{3)} We release the speech dataset, roughly 200 hours of synthetic audio data and 1 hour of real speech data, to the public.~\footnote{http://data.vision.ee.ethz.ch/daid/VQA/SpeechVQA.zip}

% To be certain, readers will notice that our models are simple. The emphasis of this paper is not on achieving the best performance for either the end-to-end nor the pipelined approach. Rather, it is to provide a baseline, to induct speech-based visual question answering into the multimedia research domain, and to provide some insights in this topic. \tz{reviewers doesn't seem to think we provided insights, perhaps rethink this paragraph}

\section{Related Works}
% \tz{remove?}
% Our work is generally relevant to visual question answering, the integration of vision and speech, and end-to-end speech recognition.

\subsection{Visual Question Answering}
The initial introduction of VQA into the AI community \cite{realtime:vqa}, \cite{daquar} was motivated by a desire to build intelligent systems that can understand the world more holistically. In order to complete the task of VQA, it was both necessary to understand a textual question and a visual scene. However, it was not until the introduction of \textit{VQA1.0} \cite{VQA} that the application took mainstream in the computer vision and natural language processing (NLP) communities.

Recently, popular topics of exploration have been on the development of attention models. Attention models were popularized by their success with the NLP community in machine translation \cite{nmt:joint:align}, and quickly demonstrated their efficacy in computer vision \cite{recurrent:visual:attn}. Within the context of visual question answering, attention mechanisms `show' a model where to look when answering a question. Stacked Attention Network \cite{vqa:stackedattn} learns an attention mechanism based on the of the question's encoding to determine the salient regions in an image. More sophisticated attention-centric models such as \cite{vqa:dualattn,vqa:hieco,vqa:dynamicmemory} were since then developed.

Other points of research are based on the pooling mechanism that combines the language component with the vision components. Some use an element-wise multiplication \cite{vqa:spatial:attn,vqa:stackedattn} to pool these modalities, while \cite{vqa:hieco} and \cite{multimodal:pooling} have shown much success in using more complex methods. Our work differs from theirs in that we do not try to improve the performance of the VQA model by adding attention mechanisms or augmenting the pooling techniques.


\subsection{Integration of Speech and Vision}
The works also relevant to ours are those integrating speech and vision. Pixeltone \cite{pixel:tone}
and Image spirit \cite{image:spirit} are examples that use voice
commands to guide image processing and semantic segmentation. There is also academic work \cite{show:tell} \cite{speech:anno:img} \cite{speech:retri:img} and an app \cite{smile} that use speech to provide image descriptions. Their tasks and algorithms are both different from ours. We study the potential of integrating speech and vision in the context of VQA and aim to learn a joint understanding of speech and vision. Those approaches, however, use speech recognition for data collection or result refinement.

Our work also shares similarity with visual-grounded speech understanding or recognition. The closest one in this vein is \cite{speech:caption}, in which a deep model is learned with speeches about image captions for speech-based image retrieval. In a broader context of integration of sound and vision, Soundnet \cite{soundnet} transfers visual information into sound representations, but this differs from our work because their end goal is to label a sound.

\subsection{End-To-End Speech Recognition}
In the past decade, deep learning has allowed many fields in artificial intelligence to replace traditional hand-crafted features and pipeline systems with end-to-end models. Since speech recognition is typically thought of as a sequence to sequence transduction problem, i.e. given an input sequence, predict an output sequence, the application of LSTM and the CTC \cite{rnn:discrimspotting,ctc} promptly showed the success needed to justify its superiority over traditional methods. Current state of the art ASR systems such as DeepSpeech2 \cite{deepspeech2} uses stacked Bi-directional Recurrent Neural Networks (RNNs) in conjunction with Convolutional Neural networks (CNNs) because these deep neural architectures all share the property of being able to perform back propagation. \tz{A few sentences about the diff between speech recognition and our task, justifying why we don't need to use the speech recognition architecture} While our work does not attempt sequence to sequence prediction nor make use of CTCs, our approach and the integration of speech with vision is nevertheless made possible by the same underlying principle of the works listed above. \tz{delete this sentence?}


% Conv dimensions table
\begin{table*}[!t]
\centering
\caption{Dimensions for the conv layers. Example shown with a 2 second long audio waveform, sampled at 16 kHz. The final output dimensions are (3, 512)}
\label{table:convdims}
\begin{tabular}{l|c|c|c|c|c|c|c|c|c}
Layer         & \multicolumn{1}{l|}{conv1} & \multicolumn{1}{l|}{pool1} & \multicolumn{1}{l|}{conv2} & \multicolumn{1}{l|}{pool2} & \multicolumn{1}{l|}{conv3} & \multicolumn{1}{l|}{pool3} & \multicolumn{1}{l|}{conv4} & \multicolumn{1}{l|}{pool4} & \multicolumn{1}{l}{conv5} \\ \hline
Input Dim     & 32,000                      & 16,000                      & 4,000                       & 2,000                       & 500                        & 250                        & 62                         & 31                         & 7                         \\
\# Filters    & 32                         & 32                         & 64                         & 64                         & 128                        & 128                        & 256                        & 256                        & 512                       \\
Filter Length & 64                         & 4                          & 32                         & 4                          & 16                         & 4                          & 8                          & 4                          & 4                         \\
Stride        & 2                          & 4                          & 2                          & 4                          & 2                          & 4                          & 2                          & 4                          & 2                         \\
Output Dim    & 16,000                      & 4,000                       & 2,000                       & 500                        & 250                        & 62                         & 31                         & 7                          & 3                        
\end{tabular}
\end{table*}


%Model Architecture
\begin{figure}[!th]
\centering
\begin{tikzpicture}[
inps/.style={rectangle, draw=black!60, very thick, minimum size=10mm},
rect/.style={rectangle, draw=black!60, very thick, minimum size=5mm},
circ/.style={circle, draw=black!60, very thick, minimum size=5mm},
empty/.style={rectangle, very thick, minimum size=5mm},
]

\node[empty]      (central)             {};

%TextMod
%Nodes
\node[circ]      (1main)          [left =15mm of central]      {.};
% Linguistic
\node[rect]      (1ldense)        [above right=3mm of 1main]        {Dense};
\node[rect]      (1lstm)     [above=2mm of 1ldense]           {LSTM};
\node[rect]      (1emb)          [above=2mm of 1lstm]      {Embedding};
\node[empty]      (1linput)    [above=1mm of 1emb]            {Text};
% Visual
\node[rect]      (1vdense)       [above left=3mm of 1main]         {Dense};
\node[empty]        (1vinput)  [above=1mm of 1vdense] {Image};
% Combined
\node[rect, outer sep=2pt]      (1cdense1)        [below=2mm of 1main]        {Dense};
\node[rect, outer sep=2pt]      (1cdense2)        [below=2mm of 1cdense1]        {Dense};
\node[empty] (1answer) [below=2mm of 1cdense2] {output};

% Lines
\draw[->] (1emb.south) -- (1lstm.north);
\draw[->] (1lstm.south) -- (1ldense.north);
\draw[->] (1ldense.south) -- (1main.east);

\draw[->] (1vdense.south) -- (1main.west);

\draw[->] (1main.south) -- (1cdense1.north);
\draw[->] (1cdense1.south) -- (1cdense2.north);
\draw[->] (1cdense2.south) -- (1answer.north);

%SpeechMod
%Nodes
\node[circ]      (2main)          [right =15mm of central]      {.};
% Linguistic
\node[rect]      (2ldense)        [above right=3mm of 2main]        {Dense};
\node[rect]      (2lstm)     [above=2mm of 2ldense]           {LSTM};
\node[rect]      (convs)          [above=2mm of 2lstm, align=center]      {Conv Layers\\(Dimensions\\in \Cref{table:convdims})};
\node[empty]    (2linput)         [above=1mm of convs]      {Audio};
% Visual
\node[rect]      (2vdense)       [above left=3mm of 2main]         {Dense};
\node[empty]     (2vinput)  [above=1mm of 2vdense] {Image};
% Combined
\node[rect, outer sep=2pt]      (2cdense1)        [below=2mm of 2main]        {Dense};
\node[rect, outer sep=2pt]      (2cdense2)        [below=2mm of 2cdense1]        {Dense};
\node[empty] (2answer) [below=2mm of 2cdense2] {output};

\draw[->] (convs.south) -- (2lstm.north);
\draw[->] (2lstm.south) -- (2ldense.north);
\draw[->] (2ldense.south) -- (2main.east);

\draw[->] (2vdense.south) -- (2main.west);

\draw[->] (2main.south) -- (2cdense1.north);
\draw[->] (2cdense1.south) -- (2cdense2.north);
\draw[->] (2cdense2.south) -- (2answer.north);
\end{tikzpicture}
\caption{TextMod (left) and SpeechMod (right) architectures}
\label{fig:modarc}
\end{figure}


\section{Model}
Two models are employed in this work, they will be referred to henceforth as TextMod and SpeechMod. TextMod and SpeechMod only differ in their language components, keeping rest of the architecture the same. On the language side, TextMod takes as input a series of one-hot encodings, followed by an embedding layer that is learned from scratch, a LSTM encoder, and a dense layer. It is similar to \textit{VQA1.0} with some minor adjustments.

The language side of SpeechMod takes as input the raw waveform, and pushes it through a series of 1D convolutions. After the CNN layers follows a LSTM. The LSTM serves the same purpose as in TextMod, which is to interpret and encode the sequence meaningfully into a single vector.

Convolution layers are used to encode waveforms because they reduce dimensionality of data while finding salient patterns. The maximum length of a spoken question in our dataset is 6.7 seconds and corresponds to a waveform length of 107,360 elements, while the minimum is 0.63 seconds and corresponds to 10,080 elements. One could directly feed the input waveform to a LSTM, but a LSTM will be unable to learn from sequences that are excessively long, so dimensionality reduction is a necessity. Each consecutive convolution layer halves in filter length but doubles the number of filters. This is done for simplicity rather than for performance optimization. The main consideration taken in choosing the parameters is that the last convolution should output dimensions of ($x$, 512), where $x$ must be a positive integer. $x$ represents the length of a sequence of 512-dim vectors. The sequence is then fed into an LSTM, which then outputs a single vector of (512). Thus, $x$ should not be too big, and the CNN parameters are chosen to ensure a sensible sequence length. The exact dimensions of the convolution layers are shown in \Cref{table:convdims}. The longest and shortest waveforms correspond to final convolution outputs of size (13, 512) and (1, 512) respectively. 512 is used as the dimension of the LSTM to be consistent with TextMod and the original VQA baseline.

On the visual side, both models ingest as input the 4,096 dimensional vector of the last layer of VGG19 \cite{vgg16} followed by a single dense layer. After both visual and linguistic representations are computed, they are merged using element-wise multiplication, a dense layer, and an output layer. The full architecture of both these models are seen in \Cref{fig:modarc}, where $\displaystyle\odot$ is the symbol for element-wise multiplication. After merging the language and visual components of each model, two dense layers are stacked. The last dense layer outputs a probability distribution over the number of output classes, and the answer corresponding to the element with the highest probability is selected.

The architectures presented in this chapter were chosen for two main reasons: simplicity and similarity. First, the intention is to keep the model complexity low. In order to establish a baseline for speech-based VQA, it is necessary to use only the bare minimum components. TextMod, as mentioned before, is similar to the original VQA baseline, which is well referenced and remains the simplest architecture on \textit{VQA1.0}. Despite its many convolution layers, SpeechMod also uses minimal components. Second, it is important that TextMod and SpeechMod differ from each other as little as possible. Similarity between models allows one to locate the source of discrepancies and helps produce a more rigorous comparison. The only difference in the two models is replacing an embedding layer with a series of convolution layers. In our implementation, the layers that are common between the two models also have same dimensions.


\section{Data}
We chose to use \textit{VQA1.0} Open-Ended dataset, for its numerous training examples and familiarity to those working in question answering. To avoid confusion, \textit{VQA1.0} henceforth refers to the dataset and the original paper, while VQA refers to the task of visual question answering. The dataset contains 248,349 questions in the training set, 121,512 in validation set, and 60,864 in the test-dev set. The complete test set contains 244,302 questions, but because the evaluation server allows for only one submission, we instead evaluate on test-dev, which has no such limit. During training, questions which do not contain the 1000 most common answers are filtered out.

Amazon Polly API is used to generate audio files for each question.~\footnote{The voice of Joanna is used: \url{https://aws.amazon.com/polly/}} The generated speech is in mp3 format, then sampled into waveform format at 16 kHz. 16 kHz was chosen due to its common usage among the speech community, but also because the model used to transcribe speech was trained on 16 kHz audio waveforms. It is worthwhile to note that the audio generator uses a female voice, thus the training and testing data are all with the same voice, except for the examples we’ve recorded, which is covered below. The full Amazon Polly speech dataset will be made available to the public.


%Diagram
\begin{figure}[t]
\centering
\begin{tikzpicture}[
scale=0.66,
every node/.style={scale=0.66},
squarednode/.style={rectangle, draw=red!70, fill=red!4, very thick, minimum size=5mm},
empty/.style={rectangle, very thick, minimum size=1mm},
]
%Nodes
\node (-center) [empty]      (model)                      {};


\node[inner sep=0.5pt, label=below:what is the number of the bus] (2n0)  [below=8mm of model] {\includegraphics[width=0.2\textwidth]{2_noise0.png}};
\node[inner sep=0.5pt, label=below:what is the number of the boss] (2n30)  [below=8mm of 2n0] {\includegraphics[width=0.2\textwidth]{2_noise30.png}};
\node[inner sep=0.5pt, label=below:where is it] (2n50)  [below=8mm of 2n30] {\includegraphics[width=0.2\textwidth]{2_noise50.png}};
\node[inner sep=0.5pt, label=below:what is the number of the boss] (2r)  [below=8mm of 2n50] {\includegraphics[width=0.2\textwidth]{2_recorded.png}};

\node[inner sep=0.5pt, label=below:what time of day is it] (1n0)  [left=4mm of 2n0] {\includegraphics[width=0.2\textwidth]{1_noise0.png}};
\node[inner sep=0.5pt, label=below:what time of day isn't] (1n30)  [left=4mm of 2n30] {\includegraphics[width=0.2\textwidth]{1_noise30.png}};
\node[inner sep=0.5pt, label=below:what time and day isn't] (1n50)  [left=4mm of 2n50] {\includegraphics[width=0.2\textwidth]{1_noise50.png}};
\node[inner sep=0.5pt, label=below:what time of day is it] (1r)  [left=4mm of 2r] {\includegraphics[width=0.2\textwidth]{1_recorded.png}};


\node[inner sep=0.5pt, label=below:are there clouds on the sky] (3n0)  [right=4mm of 2n0] {\includegraphics[width=0.2\textwidth]{3_noise0.png}};
\node[inner sep=0.5pt, label=below:are there clouds on sky] (3n30)  [right=4mm of 2n30] {\includegraphics[width=0.2\textwidth]{3_noise30.png}};
\node[inner sep=0.5pt, label=below:are there files on the sky] (3n50)  [right=4mm of 2n50] {\includegraphics[width=0.2\textwidth]{3_noise50.png}};
\node[inner sep=0.5pt, label=below:are there clouds on the sky] (3r)  [right=4mm of 2r] {\includegraphics[width=0.2\textwidth]{3_recorded.png}};

\node (-center) [empty, label=\textbf{0\% Noise}]  (n0label)  [left=2mm of 1n0]     {};
\node (-center) [empty, label=\textbf{30\% Noise}] (n30label) [left=2mm of 1n30]    {};
\node (-center) [empty, label=\textbf{50\% Noise}] (n50label) [left=2mm of 1n50]    {};
\node (-center) [empty, label=\textbf{Human}]      (hmlabel)  [left=2mm of 1r]      {};

\draw (-2.3,-1) -- (-2.3,-17);
\draw (2.3,-1) -- (2.3,-17);
\end{tikzpicture}
\caption{Spectrograms for 3 example questions with corresponding transcribed text below. 3 synthetically generated and 1 human-recorded audio clips for each question.}
\label{fig:specs}
\end{figure}


The noise we mixed with the original speech files is selected randomly from the Urban8K dataset \cite{urban8k}. This dataset contains 10 categories: air conditioner, car horn, children playing, dog bark, drilling, engine idling, gun shot, jackhammer, siren, and street music. Some clips are soft enough in volume and thus considered background noise, others are loud enough to be considered foreground noise. For each original audio file, a random noise file is selected, and combined to produce a corrupted question file according to the weighting scheme:
\begin{displaymath}
  W_{corrupted} = (1-\mathit{NL})*W_{original}  + \mathit{NL}*W_{noise}
\end{displaymath}
where \textit{NL} is the noise level. The noise audio files are subsampled to 16 kHz in order to match that of the original audio file, and is clipped to also match the spoken question length. When the spoken question is longer than the noise file, the noise file is repeated until its duration exceeds that of the spoken question. Both files are normalized before being combined so that contributions are strictly proportional to the noise level chosen. We choose 5 noise levels to mix together: 10\%-50\%, at 10\% intervals. Anything beyond 50\% is unrealistic. A visualization of different noise levels can be seen in \Cref{fig:specs} and its corresponding audio clips can be found online.\footnote{https://soundcloud.com/sbvqa/sets/speechvqa}

We also make an additional, supplementary study of the practicality of speech-based VQA with real data. 1000 questions from the \textit{val} set were randomly selected and recorded with human speakers. Two speakers (one male and one female) participated the recording task. In total, 1/3 of the data is from a male speaker, the rest is from a female speaker. Both speakers are graduate students who are not native anglophones. The data was recorded in an office environment, and there are various background noises in the audio clips as they naturally occurred.


% WER
\begin{table}[t]
\centering
\caption{Word Error Rate from Kaldi speech recognition}
\label{table:wer}
\begin{tabular}{c|c}
Noise (\%)   & WER (\%) \\ \hline
0  & 8.46  \\
10 & 12.37 \\
20 & 17.77 \\
30 & 25.41 \\
40 & 35.15 \\
50 & 47.90
\end{tabular}
\end{table}


\section{Experiments}
\subsection{Preprocessing}
For SpeechMod, the first preprocessing step is to scale each waveform to a range of [-256, 256], similar to the procedure from SoundNet \cite{soundnet}. There was no need to center each example around 0, as they are already centered. Next, each batch of waveforms were padded with 0 at the end to be of the same length.

For TextMod, the standard preprocessing steps from \textit{VQA1.0} were followed. The procedure tokenizes each sentence and replaces it with a number that corresponds to the word's index. These number indices are used as input, since the question will be fed to the model as a sequence of one hot encodings. Because questions have different lengths, the 0 index is used as padding for sequences that are too short. The 0 index essentially causes the model to skip that position. 0 is also used for unseen tokens, which is especially useful when dealing with out of vocabulary words during evaluation.

\subsection{ASR}
We use Kaldi \cite{kaldi} for ASR, due to its open-source codebase and popularity with the speech research community. The model used in this work is a DNN-HMM\footnote{https://github.com/api-ai/api-ai-english-asr-model} that has been pre-trained on assistant.ai logs (essentially short commands), making it suitable for transcribing short utterances such as the questions in \textit{VQA1.0}. Other ASRs such as wit.ai from Facebook, Cloud Speech from Google, and Bing Speech Microsoft were tested but not used in the final experiments because Kaldi achieved the lowest word error rates.


\begin{table}[t]
\centering
\caption{Accuracy on \textit{test-dev} with different levels of noise added. (Higher is better)}
\label{table:vqa test-dev}
\begin{tabular}{ll|cccc}
          &                     & All    & Y/N    & Number & Other \\ \hline
Baseline  &                     & 53.74  & 78.94  & 35.24  & 36.42 \\ \hline
TextMod   & Blind               & 48.76  & 78.20  & 35.68  & 26.59 \\
          & OQ                  & 56.66  & 78,89  & 37.24  & 42.07 \\
          & 0\%                 & 54.03  & 75.47  & 36.82  & 39.62 \\
          & 10\%                & 52.56  & 74.06  & 36.50  & 37.85 \\
          & 20\%                & 50.22  & 71.16  & 35.72  & 35.64 \\
          & 30\%                & 47.03  & 67.31  & 34.45  & 32.56 \\
          & 40\%                & 42.83  & 62.35  & 31.97  & 28.64 \\
          & 50\%                & 37.12  & 25.42  & 27.05  & 23.77 \\ \hline
SpeechMod & Blind               & 42.05  & 70.85  & 31.62  & 19.84 \\ 
          & 0\%                 & 46.99  & 67.87  & 30.84  & 32.82 \\
          & 10\%                & 45.81  & 67.29  & 30.13  & 31.03 \\
          & 20\%                & 43.33  & 65.88  & 29.24  & 27.28 \\
          & 30\%                & 40.07  & 64.15  & 27.82  & 22.28 \\
          & 40\%                & 35.85  & 61.47  & 24.68  & 16.52 \\
          & 50\%                & 32.14  & 59.33  & 20.84  & 11.50 
\end{tabular}
\end{table}


Word error rate (WER) is used to measure the accuracy of speech to text transcriptions. WER is defined as follows: 
\begin{displaymath}
  \mathit{WER} = (S+D+I)/N
\end{displaymath}
Where \textit{S} is the number of substitutions, \textit{D} is the number of deletions, and \textit{I} is the number of insertions. \textit{N} is the total number of words in the sentence being translated. Each transcribed question is compared with the original; the results are shown in \Cref{table:wer}. WER is not expected to be a perfect measure of transcription accuracy, since some words are more essential to the meaning of a sentence than other words. For example, missing the word \emph{dog} in the sentence \emph{what is the dog eating} is more detrimental than missing the word \emph{the}, but we nevertheless employ it to convey a general notion of how many words are understood by the ASR. Naturally the more noise there is, the higher the word error rate becomes. Due to transcription errors, there are resulting questions that contain words not seen in the original datasets. These words, as mentioned above, are indexed as 0 and are masked when fed into TextMod.


\subsection{Implementation}
Keras was used to run all experiments, with the adam optimizer for both architectures. No parameter tuning was done; the default adam parameters are as follows: learning rate=0.001, beta1=0.9, beta2=0.999, epsilon=1e-08, learning  rate decay=0.0. Training TextMod for 10 epochs on \textit{train} + \textit{val} takes roughly an hour on a Nvidia Titan X GPU, and our best model was taken at 30 epochs. Training SpeechMod for 10 epochs takes roughly 7 hours. The reported model is taken at 30 epochs. The code is available to the public.\footnote{Will be provided after review, omitted to maintain anonymity}


%%%%%
\section{Results}
The goal of the main experiments were to observe how each model performs with and without different levels of noise added. Results are reported on \textit{test-dev}, which corresponds to training on \textit{train} + \textit{val} (\Cref{table:vqa test-dev}). The standard format of reporting results from \textit{VQA1.0} is followed: \textit{All} is the overall accuracy, \textit{Y/N} is for questions with yes or no as answers, \textit{Number} is for questions that are answered by counting, and \textit{Other} covers the rest.



TextMod is trained on the original questions (\textit{OQ}), with the best performing model being selected. ASR is used on the 0-50\% variants to convert the audio question to text. Then, the selected model from \textit{OQ} is used to evaluate based on the transcribed text. Concretely, the best performing model obtained on \textit{test-dev} is used to evaluate the transcribed variants of \textit{test-dev}. Likewise, SpeechMod is first trained on audio data with 0\% noise, with the strongest model being selected. The selected model is used to evaluate on the 10-50\% variants of the same data subset. Typically, the best model on \textit{val} is used to evaluate on \textit{test} or another `unseen' portion of the dataset. However in these experiments, the noisy variants of the same datasets are in fact unseen because the data for which the model is trained on contains no noise. We show this in the zero-shot section of the paper.


\pgfplotsset{width=8cm,compat=1.9}
\begin{figure}[t]
\centering
\begin{tikzpicture}
\begin{axis}[
    title={},
    xlabel={Noise (\%)},
    ylabel={Accuracy (\%)},
    xmin=0, xmax=50,
    ymin=30, ymax=60,
    xtick={0,10,20,30,40,50},
    ytick={30,40,50,60},
    legend pos=north east,
    legend style={font=\fontsize{6}{6}\selectfont}
]
\addplot[
    color=blue,
    mark=square,
    ]
    coordinates {
    (0,46.99)(10,45.81)(20,43.33)(30,40.07)(40,35.85)(50,32.14)
    };
\addplot[
    color=magenta,
    mark=halfcircle
    ]
    coordinates {
    (0,54.03)(10,52.56)(20,50.22)(30,47.03)(40,42.83)(50,37.12)
    };
\addplot[mark=none, color=blue, dashed] coordinates{(0,42.05) (50,42.05)};
\addplot[mark=none, color=magenta, dashed] coordinates{(0,48.76) (50,48.76)};
\legend{SpeechMod, TextMod, SpeechMod Blind 0\% Noise, TextMod Blind \textit{OQ}}
\end{axis}
\end{tikzpicture}
\caption{SpeechMod and TextMod performance with varying amounts of added noise on \textit{test-dev}. \textit{Blind} counterparts are not tested on different noise levels.}
\label{fig:noiseplots}
\end{figure}


\textit{Blind} denotes no visual information, meaning it removes the visual components while rest of the model stays the same. TextMod \textit{Blind} is trained and evaluated on the original questions. SpeechMod \textit{Blind} is trained and evaluated on the 0\% noise audio. \textit{Baseline} is from \textit{VQA1.0} using the model `LSTM Q+I'.

A graphical version of the table is shown in \Cref{fig:noiseplots}. The constant values of SpeechMod \textit{Blind} and TextMod \textit{Blind} are included to show the noise level at which they perform better than their full model counterparts. Examples of the two models answering questions from the dataset are shown in \Cref{fig:visual:examples}.

% To logically reason about the results, I make the assumption that spoken language as a modality contains more information than written language. When going from a mode of high information to one of lesser (i.e. high-dimensional spaces to low-dimensional spaces), information must be lost, or be preserved at best. Conversely, when transforming from a mode to another richer in information, information can be preserved, but cannot be gained. Thus, running TextMod with \textit{OQ} can be understood as the upper bound of all the models. Transformations from the original, clean text to speech can at best, only preserve information. As confirmed by the results, there are no results that are better than TextMod with \textit{OQ}. When the original textual inputs are replaced with transcribed inputs from an ASR, the performance worsens as adding more noise obfuscates the system's linguistic understanding. \tz{be careful what's claimed here}

% More interesting is to compare the performance of TextMod and SpeechMod while treating audio as the starting point instead of the original text. 
One might imagine SpeechMod to perform better because of its direct optimization and end-to-end training solely for the task, yet this hypothesis does not hold true. At 0\% noise, TextMod achieves 7\% higher accuracy than SpeechMod. As noise is added, both models initially falter at similar rates, although their trends seem to head towards convergence. This is expected since, since at 100\% noise the question would not be audible at all; it would be random guessing, thus both methods would perform exactly the same.


Next, we compare TextMod and SpeechMod against their respective \textit{Blind} models. The bias of questions in \textit{VQA1.0} is well documented. Namely, if the question is understood, there is a good chance of answering correctly without looking at the image (i.e. blind guessing). For example, \textit{Y/N} questions have the answer \emph{yes} more commonly than \emph{no}, so the system should guess \emph{yes} if a question is identified to be a \textit{Y/N} type. As a reference, always answering \emph{yes} yields a \textit{Y/N} accuracy of 70.81\% on \textit{test-dev}. The bias is clearly evident in both \textit{test-dev} and \textit{val} for TextMod and SpeechMod; the \textit{Y/N} section of \textit{Blind} always performs better than that of the 0\% data. Therefore, \textit{Blind} tells us how many questions are understood by these two modes of linguistic inputs. When comparing the linguistic only models with their complementary TextMod and SpeechMod, one can be certain that performances falling below the linguistic signifies that the model no longer understands the questions. Furthermore, perceiving the image and a noisy question becomes less informative than understanding a clean question without an image.


\begin{table}[t]
\centering
\caption{Accuracy on \textit{zero-shot} with different levels of noise added. (Higher is better)}
\label{table:zs}
\begin{tabular}{ll|cccc}
          &                     & All    & Y/N    & Number & Other \\ \hline
TextMod   & OQ                  & 49.41  & 77.23  & 31.18  & 27.12 \\
          & 0\%                 & 46.41  & 73.37  & 30.64  & 24.38 \\
          & 10\%                & 45.23  & 71.93  & 30.32  & 23.24 \\
          & 20\%                & 43.30  & 69.26  & 29.63  & 21.75 \\
          & 30\%                & 40.85  & 65.84  & 28.55  & 19.89 \\
          & 40\%                & 37.79  & 62.56  & 26.10  & 16.91 \\
          & 50\%                & 34.41  & 59.58  & 21.50  & 13.42 \\ \hline
SpeechMod & 0\%                 & 37.01  & 65.58  & 23.19  & 12.99 \\
          & 10\%                & 36.52  & 65.12  & 22.83  & 12.45 \\
          & 20\%                & 35.47  & 64.04  & 22.29  & 11.29 \\
          & 30\%                & 34.08  & 62.77  & 21.45  &  9.67 \\
          & 40\%                & 32.12  & 60.59  & 19.94  &  7.81 \\
          & 50\%                & 29.88  & 57.70  & 18.20  &  6.09 
\end{tabular}
\end{table}


\subsection{Zero-Shot}
In this section zero-shot (ZS) results are analyzed to further understand the behavior of both models. ZS in the context of VQA refers to questions that were never seen in training. To get ZS data, we discard questions in \textit{val} subset that appeared in the \textit{train} subset, which decreased the number of valid questions from 104,654 to 65,365. Put differently, ZS is simply a subset of \textit{val}.

The models were trained on the complete \textit{train}, and best performing on the complete \textit{val} were selected. Next, the models were tested on the original and ZS datasets with noise injected (\Cref{table:zs}, \Cref{fig:noiseplots-zs}). These experiments were not be performed on \textit{test-dev} because the ground truth from \textit{test-dev} and \textit{test} are withheld and cannot be evaluated partially on the server.


\pgfplotsset{width=8cm,compat=1.9}
\begin{figure}[t]
\centering
\begin{tikzpicture}
\begin{axis}[
    title={},
    xlabel={Noise (\%)},
    ylabel={Accuracy (\%)},
    xmin=0, xmax=50,
    ymin=30, ymax=60,
    xtick={0,10,20,30,40,50},
    ytick={30,40,50,60},
    legend pos=north east,
    legend style={font=\fontsize{6}{6}\selectfont}
]

\addplot[
    color=blue,
    mark=none,
    dashed
    ]
    coordinates {
    ( 0,44.51)
    (10,43.62)
    (20,41.37)
    (30,37.95)
    (40,34.24)
    (50,30.54)
    };
\addplot[
    color=magenta,
    mark=none,
    dashed
    ]
    coordinates {
    ( 0,50.69)
    (10,49.28)
    (20,47.06)
    (30,43.99)
    (40,40.06)
    (50,35.54)
    };
\addplot[
    color=blue,
    mark=square
    ]
    coordinates {
    ( 0,37.01)
    (10,36.52)
    (20,35.47)
    (30,34.08)
    (40,32.12)
    (50,29.88)
    };
\addplot[
    color=magenta,
    mark=halfcircle
    ]
    coordinates {
    ( 0,46.41)
    (10,45.23)
    (20,43.30)
    (30,40.85)
    (40,37.79)
    (50,34.41)
    };
\legend{SpeechMod \textit{val}, TextMod \textit{val}, SpeechMod ZS, TextMod ZS}
\end{axis}
\end{tikzpicture}
\caption{SpeechMod and TextMod performance with varying amounts of added noise for ZS subset in reference to the complete datasets.}
\label{fig:noiseplots-zs}
\end{figure}


As one would expect, ZS accuracies are worse than accuracies of the entire set, since models tend to perform more poorly on unseen data. TextMod performs 4\% better on the complete dataset than on the ZS, and SpeechMod on the complete dataset performs better by 7\%. The performance gap decreases as more noise is added. At 50\% noise, the performance on ZS and the original data have practically converged for both models. To the models, questions seen during the training but with high amount of noise added are as foreign as unseen questions. \tz{explain why textmod does better on ZS?}

% While \textit{Y/N} and \textit{Number} type questions saw small decreases in accuracies, \textit{Other} type questions showed a 12\% decrease in TextMod and 17\% decrease in SpeechMod. This is understandable, because for \textit{Y/N} or \textit{Number} questions, there are fewer potential answers to choose from. For \textit{Y/N} questions, as long as `does ... ?', or `is ... ?' is understood, the possible answers are already limited to two. \textit{Other} type questions are answered by a wider range of responses, leading to lower probability in blind guessing answers correctly.



% Human recorded questions
\begin{table}[t]
\centering
\caption{Performance on 1000 human recorded questions.}
\label{table:recorded}
\begin{tabular}{l|cccc}
                        & All   & Y/N   & Number & Other \\ \hline
SpeechMod (Recorded)    & 21.46 & 57.26 & 0.77   & 0.91  \\
SpeechMod (Synthetic)   & 42.69 & 66.58 & 32.31  & 27.58 \\ \hline
TextMod (Recorded)      & 41.66 & 66.33 & 35.69  & 25.37 \\
TextMod (Original Text) & 53.09 & 77.73 & 41.54  & 38.26
\end{tabular}
\end{table}


\subsection{Human Recordings}
Finally, a small, supplementary test is run on non-synthetic, human recorded questions to see if the models would perform differently on real-world audio inputs. 1000 samples were randomly selected from \textit{val}, and the best performing models from the ZS section were used for evaluation. \Cref{table:recorded} shows the performance on the synthetic and human recorded versions of this subset.


\begin{figure*}[t]
$\begin{tabular}{cccc}
\includegraphics[width=0.24\linewidth, height=30mm]{images/howmany.jpg}
& \hspace{-3mm}
\includegraphics[width=0.24\linewidth, height=30mm]{images/lean.jpg}
& \hspace{-3mm}
\includegraphics[width=0.24\linewidth, height=30mm]{images/table.jpg}
& \hspace{-3mm}
\includegraphics[width=0.24\linewidth, height=30mm]{images/teddy.jpg}

\\
\vspace{-3mm}
\begin{tcolorbox}[width=0.24\linewidth, left=1pt,right=1pt,top=0pt,bottom=0pt]
how many people are in this photo? 
\end{tcolorbox}
%}
& \hspace{-3mm}
\begin{tcolorbox}[width=0.24\linewidth, left=1pt,right=1pt,top=0pt,bottom=0pt]
what is leaning against the house?
\end{tcolorbox}
%}
& \hspace{-3mm}
%\multirow{ 2}{*}{
\begin{tcolorbox}[width=0.24\linewidth, left=1pt,right=1pt,top=0pt,bottom=0pt]
what has been upcycled to make lights?
\end{tcolorbox} 
& \hspace{-3mm}
\begin{tcolorbox}[width=0.24\linewidth, left=1pt,right=1pt,top=0pt,bottom=0pt]
what is the teddy bear sitting on? 
\end{tcolorbox}  \\

\vspace{-5mm}
\\
\begin{tcolorbox}[width=0.24\linewidth, left=1pt,right=1pt,top=0pt,bottom=0pt]
 TextMod \ityping: {\color{blue} 2} \\
 SpeechMod \ispeaking: {\color{blue} 2} 
\end{tcolorbox}
%}
& \hspace{-3mm}
\begin{tcolorbox}[width=0.24\linewidth, left=1pt,right=1pt,top=0pt,bottom=0pt]
TextMod \ityping: {\color{red} tree} \\
SpeechMod \ispeaking: {\color{red} chair} 
\end{tcolorbox}
%}
& \hspace{-3mm}
%\multirow{ 2}{*}{
\begin{tcolorbox}[width=0.24\linewidth, left=1pt,right=1pt,top=0pt,bottom=0pt]
TextMod \ityping: {\color{red} bulb} \\
SpeechMod \ispeaking: {\color{red} bulb} 
\end{tcolorbox} 
& \hspace{-3mm}
\begin{tcolorbox}[width=0.24\linewidth, left=1pt,right=1pt,top=0pt,bottom=0pt]
TextMod \ityping: {\color{blue} chair} \\
SpeechMod \ispeaking: {\color{red} yes} 
\end{tcolorbox}  \\
\end{tabular}$
\vspace{-4mm}
\caption{Example of speech-based VQA answers. Correct answers in blue and incorrect in red.}  
\label{fig:visual:examples} 
\end{figure*}


Although it is clear that both models have difficulties handling recorded questions, SpeechMod performs especially poorly. TextMod on the synthetic dataset achieves similar accuracy as it does on \textit{val} and \textit{test-dev} with 40\% noise. SpeechMod however, gets similar performance as the synthetic data with 50\% noise on only the \textit{Y/N} questions, while it seems to understand none of the other question types.


\subsection{Discussion}
As a modality, speech contains more information than text. In the process of reducing the high-dimensional audio inputs to the low-dimensional class output label (i.e. the answer), the best performing system must be the one that extracts patterns most effectively (provided there are such patterns in the data). TextMod relies heavily on the intermediate ASR system, which is more complicated than the entire architecture of SpeechMod, as the number of parameters one needs to learn for speech recognition is also much greater. The Kaldi ASR has been been trained on many times more data than contained in \textit{VQA1.0}. The ASR serves to filter out noise in high dimensions and extract meaningful patterns in the form of text. In that sense, one can think of text as an explicit intermediate standardization of data before the question answering module.

Vice versa, SpeechMod does not contain explicit intermediate data standardization, so the model may not learn to extract the concept of words from audio sounds. Whether or not humans understand speech by first transforming it to words then interpreting the meaning of words is debatable. I am of the opinion that at the very least, humans first transform speech into composable concepts, whether the concepts are in the form of words, clauses, or perhaps other latent structures. An intuitive example of this is how toddlers first learn to speak. They initially speak in short utterances, gradually build up to babble incoherent sentences, and lastly learn the grammatical structure of language. Unfortunately, it is not possible to meaningfully examine the patterns learned by SpeechMod. For visual pattern recognition, 2-D CNN layers can be easily and intuitively inspected via their weights, but there are no reliable methods (at the time of writing) to visualize 1-D CNN layers when the inputs are waveforms. In summary, from the results we conclude that indirect as it may be, using ASR to obtain text first is more useful and informative than the end-to-end approach. \tz{work on this paragraph, maybe it can be combined with the following}


It is worth noting that in contrast to the characteristics of the pipelined approach, SpeechMod does not include mechanisms that forces it to learn semantics in language, and the only data it learns from is the given dataset. \tz{Change it is worth noting, this paragraph sounds a bit discussion-like} Bearing in mind its simple architecture and limited amount of data, the performance gap of 7\% between SpeechMod and TextMod is within acceptable limits and merits further study into the end-to-end methods.

\tz{ZS reason why textmod is better?}

Despite its limited sample size, the human-recorded dataset provides a few insights. The synthetic audio sounds monotonous, disinterested, with little silence in between words while the human recorded audio has inflections, emphasis, accents, and pauses. An inspection of the spectrograms (\Cref{fig:specs}) confirms this, as the synthetic waveforms contain have vastly different audio signatures. Because SpeechMod uses the waveforms as input directly and it has no training data similar to the human recorded samples, during prediction time it is unable to find salient patterns in the audio clips. By comparison, the pipelined approach has an ASR that is already trained with data containing lots of variance, so it is able to standardize the audio into a compact, salient textual representation before feeding it into TextMod. From the perspective of the text-based question answering system, its linguistic input is only slightly different than the original text it was trained on.

\tz{the same VQA paper is cited three times!!!}


\subsection{Future Work}

One straightforward approach to improving the end-to-end model is by data augmentation. It is widely accepted that effectiveness of neural architectures is data driven, so training with noisy data and different speakers will make the model more robust to inputs during run time. Just as many possibilities exist in improving the architecture. One can add feature extractors, attention mechanisms, or any amalgamation of the techniques in the deep learning mainstream.

\tz{multiple speakers, more data}

\section{Conclusion}

We propose speech-based visual question answering and introduce two approaches that tackle this problem, one of which can be trained end-to-end on audio inputs. Despite its simple architecture, the end-to-end method works well when the test data has audio signatures comparable to its training data. Both methods suffer performance decreases at similar rates when noise is introduced. A pipelined method using an ASR tolerates varied inputs much better because it normalizes the input variance into text before running the VQA module. We release the speech dataset and invite the multimedia research community to explore the intersection of speech and vision.

\newpage
\bibliographystyle{aaai}
\bibliography{sigproc} 

\end{document}
